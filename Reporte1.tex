\documentclass[12pt]{article}
\usepackage{graphicx}
\begin{document}
\maketitle

\title{\scalebox{3}{\fbox{\fbox{\centering Proyecto BABYPLAY}}}}

\begin{figure*}[htb]
\centering
\includegraphics[width=0.4\textwidth]{./logobabyplay}
\end{figure*}
\author{\centering\scshape
        Integrantes\\
        Jimmy Banchon\\
        Vanessa Revelo\\
        Stephany Samaniego\\}}
        

\section{\scalebox{1.5}{\fbox{Introducci\'on}}}
El proyecto \underline{BABYPLAY}, se basa de un juego did\'actico para ni\~nos\\
de una edad enter [2 a 4] a\~nos de edad, el beneficio de este proyecto es la estimulaci\'on
temprana.
Entendemos los Juegos como una manifestaci\'on de una actividad f\'isica, m\'as o menos reglamentada,
en donde es necesaria la creaci\'on de una estrategia para su desarrollo y para resolver los problemas que se originen. 
Desde la perspectiva de la Pedagog\'ia Activa, el Juego cumple los requisitos de participaci\'on activa y consciente,
promueve la experimentaci\'on y el razonamiento. Otro punto a su favor es el inter\'es y motivaci\'on espont\'anea que produce 
su ejecuci\'on entre los ni\~nos y ni\~nas.

\paragraph{Justificaci\'on}
Con la realizaci\'on de la propuesta en este proyecto se contribuye a facilitar
al docente un comperdio de informaci\'on sobre las visrtudes del juego como 
herramienta para desarrollar nuevos conocimientos en el ni\~no en la etapa 
de educaci\'on inicial, lo que permite garantizar el buen progreso de 
aprendizajes, y donde se establecen el uso de estrateg\'ias puntuales que 
promueven el progreso de aprendizajes, y donde se establecen el uso de 
estrategias puntuales que promueven el progreso cognitivo de un ni\~no y ni\~na,
con el apoyo de material did\'actico el cual es puesto como interfaz e un dispositivo
con sistema ANDROID, para que sea usado por aquel docerte que tome inter\'es en 
amplozar nociones sobre el tema y parta que el ni\~no refleja su comodidad en 
las habilidades creativas mas no en las te\'oricas.

\section{\scalebox{1.5}{\fbox{Objetivos}}}
\paragraph{BabyPlay 1.-}
Innovar en los juegos del Sistema Operativo \underline{Android} para estimular
el aprendizaje en los ni�os de 2 a 4 a\~nos.

\paragraph{BabyPlay 2.-}
Crear otra alternativa bajo el concepto jugando se aprende.

\paragraph{BabyPlay 3.-}
Lograr una exitosa interacci\'on entre los ni\~nos y la aplicaci\'on .

\paragraph{BabyPlay 4.-}
Reforzar a trav\'es de juegos did\'acticos el desarrollo cognitivo del ni\~no y ni\~na.

\paragraph{BabyPlay 5.-}
Desarrollar los sentidos cognitivos del ni\~no y ni\~na por mdeio de los juegso did\'acticos presentados posteriormente, en las que se utilizan imagenes que promuevan sus habilidades.


\section{\scalebox{1.5}{\fbox{Funcionalidades}}}
\paragraph{Las Funciones Cognitivas}
Las funciones cognitivas son aquellas que nos permiten efectuar actividades como
elaborar acciones de rutina, recordar un animal, reconocer alguna imagen vista o 
simplemente leer.

\paragraph{Atenci\'on}
La atenci\'on es un proceso cognitivo en que el sujeto selecciona la informaci\on y procesa
solo algunos datos entre la m\'utiple estimulaci\'on sensorial.

\section{\scalebox{1.5}{\fbox{Detalles}}}
\begin{figure}[h!]
\begin{minipage}{0.5\textwidth}
\centering
 \includegraphics[width=1\textwidth]{./1p}
\end{minipage}
\hfill\begin{minipage}{0.5\textwidth} Este es un prototipo de como seria la interfaz de nuestro proyecto. Aqui mostramos de como seria la interfaz cuando el ni�o escoga la opcion de {\sc Animales} en la que le saldra la silueta.
\end{minipage}
\end{figure}

\begin{figure}[h]
\begin{minipage}{0.5 \textwidth}
Aqui mostramos las opciones que le saldran y como estar\'an distribuidas.
\end{minipage}
\hfill \begin{minipage}{6.5cm}
\begin{center}
 \includegraphics[width=1\textwidth]{./2p}
\end{center}
\end{minipage}
\end{figure}

\begin{figure}[h!]
\begin{minipage}{0.5\textwidth}
\centering
 \includegraphics[width=1\textwidth]{./3p}
\end{minipage}
\hfill\begin{minipage}{0.5\textwidth} Y como resultado se tendr\'a la siguiente p\'agina.
\end{minipage}
\end{figure}

\begin{figure}[h!]
\begin{minipage}{0.5 \textwidth}
En esta parte mostramos la secci\'on de rompecabezas que utiliza la camara para generarlo.
\end{minipage}
\hfill \begin{minipage}{6.5cm}
\begin{center}
 \includegraphics[width=1\textwidth]{./4p}
\end{center}
\end{minipage}
\end{figure}



\end{document}
