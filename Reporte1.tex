\documentclass[12pt]{article}
\usepackage{graphicx}
\begin{document}
\maketitle

\title{\scalebox{3}{\fbox{\fbox{\centering Proyecto BABYPLAY}}}}

\begin{figure*}[htb]
\centering
\includegraphics[width=0.4\textwidth]{./logobabyplay}
\end{figure*}
\author{\centering\scshape
        Integrantes\\
        Jimmy Banchon\\
        Vanessa Revelo\\
        Stephany Samaniego\\}}
        

\section{\scalebox{1.5}{\fbox{Introducci\'on}}}
El proyecto \underline{BABYPLAY}, se basa de un juego did\'actico para ni\~nos\\
de una edad enter [2 a 4] a\~nos de edad, el beneficio de este proyecto es la estimulaci\'on
temprana.
Entendemos los Juegos como una manifestaci\'on de una actividad f\'isica, m\'as o menos reglamentada,
en donde es necesaria la creaci\'on de una estrategia para su desarrollo y para resolver los problemas que se originen. 
Desde la perspectiva de la Pedagog\'ia Activa, el Juego cumple los requisitos de participaci\'on activa y consciente,
promueve la experimentaci\'on y el razonamiento. Otro punto a su favor es el inter\'es y motivaci\'on espont\'anea que produce 
su ejecuci\'on entre los ni\~nos y ni\~nas.

\paragraph{Justificaci\'on}
Con la realizaci\'on de la propuesta en este proyecto se contribuye a facilitar
al docente un comperdio de informaci\'on sobre las visrtudes del juego como 
herramienta para desarrollar nuevos conocimientos en el ni\~no en la etapa 
de educaci\'on inicial, lo que permite garantizar el buen progreso de 
aprendizajes, y donde se establecen el uso de estrateg\'ias puntuales que 
promueven el progreso de aprendizajes, y donde se establecen el uso de 
estrategias puntuales que promueven el progreso cognitivo de un ni\~no y ni\~na,
con el apoyo de material did\'actico el cual es puesto como interfaz e un dispositivo
con sistema ANDROID, para que sea usado por aquel docerte que tome inter\'es en 
amplozar nociones sobre el tema y parta que el ni\~no refleja su comodidad en 
las habilidades creativas mas no en las te\'oricas.

\section{\scalebox{1.5}{\fbox{Detalles}}}
\begin{figure}[h!]
\begin{minipage}{0.5\textwidth}
\centering
 \includegraphics[width=1\textwidth]{./1p}
\end{minipage}
\hfill\begin{minipage}{0.5\textwidth} Este es un prototipo de como seria la interfaz de nuestro proyecto. Aqui mostramos de como seria la interfaz cuando el ni�o escoga la opcion de {\sc Animales} en la que le saldra la silueta.
\end{minipage}
\end{figure}

\begin{figure}[h]
\begin{minipage}{0.5 \textwidth}
Aqui mostramos las opciones que le saldran y como estar\'an distribuidas.
\end{minipage}
\hfill \begin{minipage}{6.5cm}
\begin{center}
 \includegraphics[width=1\textwidth]{./2p}
\end{center}
\end{minipage}
\end{figure}

\begin{figure}[h!]
\begin{minipage}{0.5\textwidth}
\centering
 \includegraphics[width=1\textwidth]{./3p}
\end{minipage}
\hfill\begin{minipage}{0.5\textwidth} Y como resultado se tendr\'a la siguiente p\'agina.
\end{minipage}
\end{figure}



\section{\scalebox{1.5}{\fbox{Experiencias Con \LaTeX{}}}}
\paragraph{Vanessa Revelo Heras}
La experiencia que tuve al utilizar latex es que para mi fue muy algo beneficioso
ya que tuve el privilegio de utilizar otro sistema de composici\'on de texto plano.
La pr\'actica con \LaTeX{} me ha ayudado a ver de otra perspectiva el uso de los
programas de texto plano como word.

\paragraph{Stephany Samaniego Villarroel}

\paragraph{Jimmy Banchon}

\section{\scalebox{1.5}{\fbox{Conclusiones}}}
Conclusiones....


\end{document}
