\documentclass[12pt]{article}
\usepackage{graphicx}
\usepackage{color}
\begin{document}
\maketitle

\title{\scalebox{3}{\fbox{\fbox{\centering Proyecto BABYPLAY}}}}


\begin{figure*}[htb]
\centering
\includegraphics[width=0.3\textwidth]{./logobabyplay}
\end{figure*}

\newpage

\begin{figure}[h!]
\begin{minipage}{0.5\textwidth}
\centering
 \includegraphics[width=2\textwidth]{./tra}
\end{minipage}
\hfill\begin{minipage}{0.5\textwidth}\color{white}INTEGRANTES:\\
Vanessa Revelo Heras\\
Stephany Samaniego Villarroel\\
Jimmy Banchon Zambrano
\end{minipage}
\end{figure}

\newpage

\chapter{\scalebox{1.5}{\fbox{INDICE}}}
\pagenumbering{arabic}
\section{Introducci\'on}
\subsection{Justificaci\'on}
\section{Observaciones}
\section{Objetivos}
\section{Funcionalidades}
\subsection{Las Funciones Cognitivas}
\subsection{Atenci\'on}
\section{Experiencia en el Desarrollo}
\section{Detalles}
\section{Conclusi\'on}


\newpage

\chapter{\scalebox{2.5}{\fbox{Introducci\'on}}}
\\\\El proyecto \underline{BABYPLAY}, se basa de un juego did\'actico para ni\~nos\\
de una edad enter [2 a 4] a\~nos de edad, el beneficio de este proyecto es la estimulaci\'on
temprana.
Entendemos los Juegos como una manifestaci\'on de una actividad f\'isica, m\'as o menos reglamentada,
en donde es necesaria la creaci\'on de una estrategia para su desarrollo y para resolver los problemas que se originen. 
Desde la perspectiva de la Pedagog\'ia Activa, el Juego cumple los requisitos de participaci\'on activa y consciente,
promueve la experimentaci\'on y el razonamiento. Otro punto a su favor es el inter\'es y motivaci\'on espont\'anea que produce 
su ejecuci\'on entre los ni\~nos y ni\~nas.

\paragraph{Justificaci\'on}
Con la realizaci\'on de la propuesta en este proyecto se contribuye a facilitar
al docente un comperdio de informaci\'on sobre las visrtudes del juego como 
herramienta para desarrollar nuevos conocimientos en el ni\~no en la etapa 
de educaci\'on inicial, lo que permite garantizar el buen progreso de 
aprendizajes, y donde se establecen el uso de estrateg\'ias puntuales que 
promueven el progreso de aprendizajes, y donde se establecen el uso de 
estrategias puntuales que promueven el progreso cognitivo de un ni\~no y ni\~na,
con el apoyo de material did\'actico el cual es puesto como interfaz e un dispositivo
con sistema ANDROID, para que sea usado por aquel docerte que tome inter\'es en 
amplozar nociones sobre el tema y parta que el ni\~no refleja su comodidad en 
las habilidades creativas mas no en las te\'oricas.\\\\

\chapter{\scalebox{2.5}{\fbox{Observaciones}}}\\
\\El juego hecho en nuestra aplicaci\'on es para ni\~nos que recien esta en la edad de aprendizaje.\\
El ni\~no deber\'a interactuar con la aplicaci\'on de manera en que comienze a aprender las diferentes clases de animales, y poderlas reconocer en el mundo real.\\
La aplicac\'on fue de gran experiencia en nuestra carrera, por el ingenio , compatibilidad y creactividad que nos toc\'o realizar como equipo.\\

\chapter{\scalebox{2.5}{\fbox{Objetivos}}}
\\\\
\paragraph{BabyPlay 1.-}
Innovar en los juegos del Sistema Operativo \underline{Android} para estimular
el aprendizaje en los ni�os de 2 a 4 a\~nos.

\paragraph{BabyPlay 2.-}
Crear otra alternativa bajo el concepto jugando se aprende.

\paragraph{BabyPlay 3.-}
Lograr una exitosa interacci\'on entre los ni\~nos y la aplicaci\'on .

\paragraph{BabyPlay 4.-}
Reforzar a trav\'es de juegos did\'acticos el desarrollo cognitivo del ni\~no y ni\~na.

\paragraph{BabyPlay 5.-}
Desarrollar los sentidos cognitivos del ni\~no y ni\~na por mdeio de los juegso did\'acticos presentados posteriormente, en las que se utilizan imagenes que promuevan sus habilidades.\\\\

\chapter{\scalebox{2.5}{\fbox{Funcionalidades}}}
\\\\
\paragraph{Las Funciones Cognitivas}
Las funciones cognitivas son aquellas que nos permiten efectuar actividades como
elaborar acciones de rutina, recordar un animal, reconocer alguna imagen vista o 
simplemente leer.

\paragraph{Atenci\'on}
La atenci\'on es un proceso cognitivo en que el sujeto selecciona la informaci\on y procesa
solo algunos datos entre la m\'utiple estimulaci\'on sensorial.\\\\

\chapter{\scalebox{2.5}{\fbox{Instrucciones}}}\\\\
1.- Escoja la Opci\'on de animales\\
2.- Al ingresar a la p\'agina podr\'a interactuar con el juego\\
3.- Escojer la opci\'on que mas le parezca\\
4.- Al acertar saldr\'a un video del animal acertado\\

\chapter{\scalebox{2.5}{\fbox{Experiencia en el Desarrollo}}}
\\\\
En nuestra experiencia al implementar cada parte, estructura de nuestra aplicaci\'on fue basada en ver e imaginarnos como podr\'ia un ni\~no aprender en forma
din\'amica, y entretenida.
Al inicio esta aplicaci\'on nos pareci\'o compleja por el motivo de que no teniamos mayores conocimientos respecto al tema. Al iniciar nuestras investigaciones, vimos que no era tan complejo aunque nos dificultaba la manera en que interactua el eclipse no nos gustaba ya que escribiamos alg\'un c\'odigo y no cog\'ia revisando el c\'odigo en otra computadora.
Creemos que nuestra aplicaci\'on es una gran iniciativa para muchos padres de familia, para un aprendizaje mayor al ni\~no.\\\\

\chapter{\scalebox{2.5}{\fbox{Detalles}}}\\\\
\begin{figure}[h!]
\begin{minipage}{0.5 \textwidth}
Al inicio de nuestra aplicaci\'on se presentar\'a la siguiente imagen
en la que se observara las siguientes opciones en la cuales el ni\~no podr\'a interactuar.
\end{minipage}
\hfill \begin{minipage}{6.5cm}
\begin{center}
 \includegraphics[width=1\textwidth]{./6p}
\end{center}
\end{minipage}
\end{figure}

\begin{figure}[h!]
\begin{minipage}{0.5\textwidth}
\centering
 \includegraphics[width=1\textwidth]{./p2}
\end{minipage}
\hfill\begin{minipage}{0.5\textwidth} Este es un prototipo de como seria la interfaz de nuestro proyecto. Aqui mostramos de como seria la interfaz cuando el ni�o escoga la opcion de {\sc Animales} en la que le saldra la silueta.
en donde los marcos representan los animales a escoger.
\end{minipage}
\end{figure}

\begin{figure}[h]
\begin{minipage}{0.5 \textwidth}
Aqui mostramos las opciones que le saldran y como estar\'an distribuidas.
\end{minipage}
\hfill \begin{minipage}{6.5cm}
\begin{center}
 \includegraphics[width=1\textwidth]{./p1}
\end{center}
\end{minipage}
\end{figure}

\begin{figure}[h!]
\begin{minipage}{0.5\textwidth}
\centering
 \includegraphics[width=1\textwidth]{./p3}
\end{minipage}
\hfill\begin{minipage}{0.5\textwidth} Y como resultado se tendr\'a la siguiente p\'agina.
\end{minipage}
\end{figure}

\chapter{\scalebox{2.5}{\fbox{Conslusi\'on}}}
\\\\
Al finalizar nuestro proyecto el juego did\'actico...\\
Fue hecho con la finalidad de entretener e incetivar al ni\~no/a a aprender las diferentes clases de una manera divertida.\\
El juego tiene un gran valor educativo para el ni\~no, porque desde el punto de vista pedag\'ogico se dice que el juego es una actividad vital espont\'anea y permanente 
del ni�o.\\
Mediante el juego y de acuerdo a una buena dosificaci\'on se descubren en los ni\~nos 
valores, aptitudes f\'isicas que posteriormente se pueden ir puliendo, perfeccionando para 
poder desembocar en una futura promesa de nuestra pr\'actica de sentimiento deportivo e intelectual.\\
El ni\~no por medio de los distintos animales realiza sucesiva identificaci�n con el mundo externo.
El juego es un medio esencial de organizaci\'on desarrollo y afirmaci\'on de la personalidad.


\end{document}
